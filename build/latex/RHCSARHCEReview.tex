% Generated by Sphinx.
\def\sphinxdocclass{report}
\documentclass[letterpaper,10pt,english]{sphinxmanual}
\usepackage[utf8]{inputenc}
\DeclareUnicodeCharacter{00A0}{\nobreakspace}
\usepackage[T1]{fontenc}
\usepackage{babel}
\usepackage{times}
\usepackage[Bjarne]{fncychap}
\usepackage{longtable}
\usepackage{sphinx}
\usepackage{multirow}


\title{RHCSA / RHCE Review Documentation}
\date{September 24, 2014}
\release{2014.09.18}
\author{Robb Hendershot}
\newcommand{\sphinxlogo}{}
\renewcommand{\releasename}{Release}
\makeindex

\makeatletter
\def\PYG@reset{\let\PYG@it=\relax \let\PYG@bf=\relax%
    \let\PYG@ul=\relax \let\PYG@tc=\relax%
    \let\PYG@bc=\relax \let\PYG@ff=\relax}
\def\PYG@tok#1{\csname PYG@tok@#1\endcsname}
\def\PYG@toks#1+{\ifx\relax#1\empty\else%
    \PYG@tok{#1}\expandafter\PYG@toks\fi}
\def\PYG@do#1{\PYG@bc{\PYG@tc{\PYG@ul{%
    \PYG@it{\PYG@bf{\PYG@ff{#1}}}}}}}
\def\PYG#1#2{\PYG@reset\PYG@toks#1+\relax+\PYG@do{#2}}

\expandafter\def\csname PYG@tok@gd\endcsname{\def\PYG@tc##1{\textcolor[rgb]{0.63,0.00,0.00}{##1}}}
\expandafter\def\csname PYG@tok@gu\endcsname{\let\PYG@bf=\textbf\def\PYG@tc##1{\textcolor[rgb]{0.50,0.00,0.50}{##1}}}
\expandafter\def\csname PYG@tok@gt\endcsname{\def\PYG@tc##1{\textcolor[rgb]{0.00,0.27,0.87}{##1}}}
\expandafter\def\csname PYG@tok@gs\endcsname{\let\PYG@bf=\textbf}
\expandafter\def\csname PYG@tok@gr\endcsname{\def\PYG@tc##1{\textcolor[rgb]{1.00,0.00,0.00}{##1}}}
\expandafter\def\csname PYG@tok@cm\endcsname{\let\PYG@it=\textit\def\PYG@tc##1{\textcolor[rgb]{0.25,0.50,0.56}{##1}}}
\expandafter\def\csname PYG@tok@vg\endcsname{\def\PYG@tc##1{\textcolor[rgb]{0.73,0.38,0.84}{##1}}}
\expandafter\def\csname PYG@tok@m\endcsname{\def\PYG@tc##1{\textcolor[rgb]{0.13,0.50,0.31}{##1}}}
\expandafter\def\csname PYG@tok@mh\endcsname{\def\PYG@tc##1{\textcolor[rgb]{0.13,0.50,0.31}{##1}}}
\expandafter\def\csname PYG@tok@cs\endcsname{\def\PYG@tc##1{\textcolor[rgb]{0.25,0.50,0.56}{##1}}\def\PYG@bc##1{\setlength{\fboxsep}{0pt}\colorbox[rgb]{1.00,0.94,0.94}{\strut ##1}}}
\expandafter\def\csname PYG@tok@ge\endcsname{\let\PYG@it=\textit}
\expandafter\def\csname PYG@tok@vc\endcsname{\def\PYG@tc##1{\textcolor[rgb]{0.73,0.38,0.84}{##1}}}
\expandafter\def\csname PYG@tok@il\endcsname{\def\PYG@tc##1{\textcolor[rgb]{0.13,0.50,0.31}{##1}}}
\expandafter\def\csname PYG@tok@go\endcsname{\def\PYG@tc##1{\textcolor[rgb]{0.20,0.20,0.20}{##1}}}
\expandafter\def\csname PYG@tok@cp\endcsname{\def\PYG@tc##1{\textcolor[rgb]{0.00,0.44,0.13}{##1}}}
\expandafter\def\csname PYG@tok@gi\endcsname{\def\PYG@tc##1{\textcolor[rgb]{0.00,0.63,0.00}{##1}}}
\expandafter\def\csname PYG@tok@gh\endcsname{\let\PYG@bf=\textbf\def\PYG@tc##1{\textcolor[rgb]{0.00,0.00,0.50}{##1}}}
\expandafter\def\csname PYG@tok@ni\endcsname{\let\PYG@bf=\textbf\def\PYG@tc##1{\textcolor[rgb]{0.84,0.33,0.22}{##1}}}
\expandafter\def\csname PYG@tok@nl\endcsname{\let\PYG@bf=\textbf\def\PYG@tc##1{\textcolor[rgb]{0.00,0.13,0.44}{##1}}}
\expandafter\def\csname PYG@tok@nn\endcsname{\let\PYG@bf=\textbf\def\PYG@tc##1{\textcolor[rgb]{0.05,0.52,0.71}{##1}}}
\expandafter\def\csname PYG@tok@no\endcsname{\def\PYG@tc##1{\textcolor[rgb]{0.38,0.68,0.84}{##1}}}
\expandafter\def\csname PYG@tok@na\endcsname{\def\PYG@tc##1{\textcolor[rgb]{0.25,0.44,0.63}{##1}}}
\expandafter\def\csname PYG@tok@nb\endcsname{\def\PYG@tc##1{\textcolor[rgb]{0.00,0.44,0.13}{##1}}}
\expandafter\def\csname PYG@tok@nc\endcsname{\let\PYG@bf=\textbf\def\PYG@tc##1{\textcolor[rgb]{0.05,0.52,0.71}{##1}}}
\expandafter\def\csname PYG@tok@nd\endcsname{\let\PYG@bf=\textbf\def\PYG@tc##1{\textcolor[rgb]{0.33,0.33,0.33}{##1}}}
\expandafter\def\csname PYG@tok@ne\endcsname{\def\PYG@tc##1{\textcolor[rgb]{0.00,0.44,0.13}{##1}}}
\expandafter\def\csname PYG@tok@nf\endcsname{\def\PYG@tc##1{\textcolor[rgb]{0.02,0.16,0.49}{##1}}}
\expandafter\def\csname PYG@tok@si\endcsname{\let\PYG@it=\textit\def\PYG@tc##1{\textcolor[rgb]{0.44,0.63,0.82}{##1}}}
\expandafter\def\csname PYG@tok@s2\endcsname{\def\PYG@tc##1{\textcolor[rgb]{0.25,0.44,0.63}{##1}}}
\expandafter\def\csname PYG@tok@vi\endcsname{\def\PYG@tc##1{\textcolor[rgb]{0.73,0.38,0.84}{##1}}}
\expandafter\def\csname PYG@tok@nt\endcsname{\let\PYG@bf=\textbf\def\PYG@tc##1{\textcolor[rgb]{0.02,0.16,0.45}{##1}}}
\expandafter\def\csname PYG@tok@nv\endcsname{\def\PYG@tc##1{\textcolor[rgb]{0.73,0.38,0.84}{##1}}}
\expandafter\def\csname PYG@tok@s1\endcsname{\def\PYG@tc##1{\textcolor[rgb]{0.25,0.44,0.63}{##1}}}
\expandafter\def\csname PYG@tok@gp\endcsname{\let\PYG@bf=\textbf\def\PYG@tc##1{\textcolor[rgb]{0.78,0.36,0.04}{##1}}}
\expandafter\def\csname PYG@tok@sh\endcsname{\def\PYG@tc##1{\textcolor[rgb]{0.25,0.44,0.63}{##1}}}
\expandafter\def\csname PYG@tok@ow\endcsname{\let\PYG@bf=\textbf\def\PYG@tc##1{\textcolor[rgb]{0.00,0.44,0.13}{##1}}}
\expandafter\def\csname PYG@tok@sx\endcsname{\def\PYG@tc##1{\textcolor[rgb]{0.78,0.36,0.04}{##1}}}
\expandafter\def\csname PYG@tok@bp\endcsname{\def\PYG@tc##1{\textcolor[rgb]{0.00,0.44,0.13}{##1}}}
\expandafter\def\csname PYG@tok@c1\endcsname{\let\PYG@it=\textit\def\PYG@tc##1{\textcolor[rgb]{0.25,0.50,0.56}{##1}}}
\expandafter\def\csname PYG@tok@kc\endcsname{\let\PYG@bf=\textbf\def\PYG@tc##1{\textcolor[rgb]{0.00,0.44,0.13}{##1}}}
\expandafter\def\csname PYG@tok@c\endcsname{\let\PYG@it=\textit\def\PYG@tc##1{\textcolor[rgb]{0.25,0.50,0.56}{##1}}}
\expandafter\def\csname PYG@tok@mf\endcsname{\def\PYG@tc##1{\textcolor[rgb]{0.13,0.50,0.31}{##1}}}
\expandafter\def\csname PYG@tok@err\endcsname{\def\PYG@bc##1{\setlength{\fboxsep}{0pt}\fcolorbox[rgb]{1.00,0.00,0.00}{1,1,1}{\strut ##1}}}
\expandafter\def\csname PYG@tok@kd\endcsname{\let\PYG@bf=\textbf\def\PYG@tc##1{\textcolor[rgb]{0.00,0.44,0.13}{##1}}}
\expandafter\def\csname PYG@tok@ss\endcsname{\def\PYG@tc##1{\textcolor[rgb]{0.32,0.47,0.09}{##1}}}
\expandafter\def\csname PYG@tok@sr\endcsname{\def\PYG@tc##1{\textcolor[rgb]{0.14,0.33,0.53}{##1}}}
\expandafter\def\csname PYG@tok@mo\endcsname{\def\PYG@tc##1{\textcolor[rgb]{0.13,0.50,0.31}{##1}}}
\expandafter\def\csname PYG@tok@mi\endcsname{\def\PYG@tc##1{\textcolor[rgb]{0.13,0.50,0.31}{##1}}}
\expandafter\def\csname PYG@tok@kn\endcsname{\let\PYG@bf=\textbf\def\PYG@tc##1{\textcolor[rgb]{0.00,0.44,0.13}{##1}}}
\expandafter\def\csname PYG@tok@o\endcsname{\def\PYG@tc##1{\textcolor[rgb]{0.40,0.40,0.40}{##1}}}
\expandafter\def\csname PYG@tok@kr\endcsname{\let\PYG@bf=\textbf\def\PYG@tc##1{\textcolor[rgb]{0.00,0.44,0.13}{##1}}}
\expandafter\def\csname PYG@tok@s\endcsname{\def\PYG@tc##1{\textcolor[rgb]{0.25,0.44,0.63}{##1}}}
\expandafter\def\csname PYG@tok@kp\endcsname{\def\PYG@tc##1{\textcolor[rgb]{0.00,0.44,0.13}{##1}}}
\expandafter\def\csname PYG@tok@w\endcsname{\def\PYG@tc##1{\textcolor[rgb]{0.73,0.73,0.73}{##1}}}
\expandafter\def\csname PYG@tok@kt\endcsname{\def\PYG@tc##1{\textcolor[rgb]{0.56,0.13,0.00}{##1}}}
\expandafter\def\csname PYG@tok@sc\endcsname{\def\PYG@tc##1{\textcolor[rgb]{0.25,0.44,0.63}{##1}}}
\expandafter\def\csname PYG@tok@sb\endcsname{\def\PYG@tc##1{\textcolor[rgb]{0.25,0.44,0.63}{##1}}}
\expandafter\def\csname PYG@tok@k\endcsname{\let\PYG@bf=\textbf\def\PYG@tc##1{\textcolor[rgb]{0.00,0.44,0.13}{##1}}}
\expandafter\def\csname PYG@tok@se\endcsname{\let\PYG@bf=\textbf\def\PYG@tc##1{\textcolor[rgb]{0.25,0.44,0.63}{##1}}}
\expandafter\def\csname PYG@tok@sd\endcsname{\let\PYG@it=\textit\def\PYG@tc##1{\textcolor[rgb]{0.25,0.44,0.63}{##1}}}

\def\PYGZbs{\char`\\}
\def\PYGZus{\char`\_}
\def\PYGZob{\char`\{}
\def\PYGZcb{\char`\}}
\def\PYGZca{\char`\^}
\def\PYGZam{\char`\&}
\def\PYGZlt{\char`\<}
\def\PYGZgt{\char`\>}
\def\PYGZsh{\char`\#}
\def\PYGZpc{\char`\%}
\def\PYGZdl{\char`\$}
\def\PYGZhy{\char`\-}
\def\PYGZsq{\char`\'}
\def\PYGZdq{\char`\"}
\def\PYGZti{\char`\~}
% for compatibility with earlier versions
\def\PYGZat{@}
\def\PYGZlb{[}
\def\PYGZrb{]}
\makeatother

\begin{document}

\maketitle
\tableofcontents
\phantomsection\label{index::doc}



\chapter{Introduction}
\label{intro:rhcsa-rhce-review}\label{intro:introduction}\label{intro::doc}

\section{The Workshop}
\label{intro:the-workshop}

\subsection{Details}
\label{intro:details}\begin{itemize}
\item {} 
The goal of the workshop is to prepare for the Red Hat Certified System Administrator and Engineer exams.

\item {} 
7pm - 9pm Thursdays.

\item {} 
Scheduled 8 weeks but we can extend it.

\end{itemize}


\subsection{Facilitator}
\label{intro:facilitator}\begin{itemize}
\item {} 
Robb Hendershot

\item {} 
\textasciitilde{}10 years as Linux Sys Admin for Lockheed Martin

\item {} 
RHCSA RHEL4/RHEL5

\item {} 
RHCSA RHEL6

\item {} 
Taking RHEL7 exams 2nd week of October

\end{itemize}


\subsection{Intros}
\label{intro:intros}\begin{itemize}
\item {} 
Name

\item {} 
Linux/Unix background

\item {} 
Why you are here

\end{itemize}


\subsection{Useful Resources}
\label{intro:useful-resources}\begin{itemize}
\item {} 
Prereq test

\item {} 
Cert Depot Study Guide - \href{http://www.certdepot.net/rhel7-rhce-exam-objectives/}{http://www.certdepot.net/rhel7-rhce-exam-objectives/}

\end{itemize}


\section{RHEL Overview}
\label{intro:rhel-overview}\begin{itemize}
\item {} 
Red Hat Enterprise Linux

\item {} 
RHEL line started in 2003

\item {} 
RHEL5 supported into 2020

\item {} 
RHEL6 supported into 2023

\item {} 
RHEL7 supported into 2027

\item {} 
Many variants including Scientific Linux and Oracle Linux

\item {} 
CentOS - Unsupported downstream version

\item {} 
Fedora - Unsupported upstream version

\end{itemize}


\section{RHCSA}
\label{intro:rhcsa}
\textbf{Red Hat Certified System Administrator (RHCSA)}

An RHCSA certification is earned when an IT professional demonstrates the core system-administration skills required in Red Hat Enterprise Linux environments.


\subsection{RHCSA Prep Classes}
\label{intro:rhcsa-prep-classes}
\textbf{For Windows Admins}
\begin{itemize}
\item {} 
Red Hat System Administration I (RH124)

\item {} 
Red Hat System Administration II (RH134)

\end{itemize}

\textbf{For Linux Admins}
\begin{itemize}
\item {} 
RHCSA Rapid Track Course (RH199)

\end{itemize}


\subsection{RHCSA Exam}
\label{intro:rhcsa-exam}
\textbf{Exam}
\begin{itemize}
\item {} 
EX200 - Red Hat Certified System Administrator (RHCSA) exam (\$400)

\end{itemize}

\textbf{Web Page}
\begin{itemize}
\item {} 
\href{http://www.redhat.com/en/services/certification/rhcsa}{http://www.redhat.com/en/services/certification/rhcsa}

\end{itemize}


\section{RHCE}
\label{intro:rhce}
\textbf{Red Hat Certified Engineer (RHCE)}

A Red Hat Certified Engineer (RHCE) is a Red Hat Certified System Administrator (RHCSA) who possesses the additional skills, knowledge, and abilities required of a senior system administrator responsible for Red Hat Enterprise Linux systems.


\subsection{RHCE Prep Classes}
\label{intro:rhce-prep-classes}
\textbf{For Windows Admins}
\begin{itemize}
\item {} 
Red Hat System Administration I (RH124)

\item {} 
Red Hat System Administration II (RH134)

\item {} 
Red Hat System Administration III (RH254)

\end{itemize}

\textbf{For Linux Admin}
\begin{itemize}
\item {} 
RHCSA Rapid Track Course (RH199)

\item {} 
Red Hat System Administration III (RH254)

\item {} 
RHCE Certification Lab (to re-certify)

\end{itemize}


\subsection{RHCE Exam}
\label{intro:rhce-exam}
\textbf{Exam}
\begin{itemize}
\item {} 
EX200 - Red Hat Certified System Administrator (RHCSA) exam (\$400)

\item {} 
EX300 - Red Hat Certified Engineer (RHCE) exam (\$400)

\end{itemize}

\textbf{Web Page}
\begin{itemize}
\item {} 
\href{http://www.redhat.com/en/services/certification/rhce}{http://www.redhat.com/en/services/certification/rhce}

\end{itemize}


\section{RHCSA Requirements (ex200)}
\label{intro:rhcsa-requirements-ex200}
Requirements for the RHCSA exam can be found at-
\href{http://www.redhat.com/en/services/training/ex200-red-hat-certified-system-administrator-rhcsa-exam}{http://www.redhat.com/en/services/training/ex200-red-hat-certified-system-administrator-rhcsa-exam}


\section{RHCSA Requirements (ex200)}
\label{intro:id1}

\subsection{Understand and use essential tools}
\label{intro:understand-and-use-essential-tools}\begin{itemize}
\item {} 
Access a shell prompt and issue commands with correct syntax

\item {} 
Use input-output redirection (\textgreater{}, \textgreater{}\textgreater{}, \textbar{}, 2\textgreater{}, etc.)

\item {} 
Use grep and regular expressions to analyze text

\item {} 
Access remote systems using ssh

\item {} 
Log in and switch users in multiuser targets

\item {} 
Archive, compress, unpack, and uncompress files using tar, star, gzip, and bzip2

\end{itemize}
\begin{itemize}
\item {} 
Create and edit text files

\item {} 
Create, delete, copy, and move files and directories

\item {} 
Create hard and soft links

\item {} 
List, set, and change standard ugo/rwx permissions

\item {} 
Locate, read, and use system documentation including man, info, and files in /usr/share/doc

\end{itemize}


\subsection{Operate running systems}
\label{intro:operate-running-systems}\begin{itemize}
\item {} 
Boot, reboot, and shut down a system normally

\item {} 
Boot systems into different targets manually

\item {} 
Interrupt the boot process in order to gain access to a system

\item {} 
Identify CPU/memory intensive processes, adjust process priority with renice, and kill processes

\item {} 
Locate and interpret system log files and journals

\item {} 
Access a virtual machine's console

\item {} 
Start and stop virtual machines

\item {} 
Start, stop, and check the status of network services

\item {} 
Securely transfer files between systems

\end{itemize}


\subsection{RHCSA Reqs - Configure local storage}
\label{intro:rhcsa-reqs-configure-local-storage}\begin{itemize}
\item {} 
List, create, delete partitions on MBR and GPT disks

\item {} 
Create and remove physical volumes, assign physical volumes to volume groups, and create and delete logical volumes

\item {} 
Configure systems to mount file systems at boot by Universally Unique ID (UUID) or label

\item {} 
Add new partitions and logical volumes, and swap to a system non-destructively

\end{itemize}


\subsection{RHCSA Reqs - Create and configure file systems}
\label{intro:rhcsa-reqs-create-and-configure-file-systems}\begin{itemize}
\item {} 
Create, mount, unmount, and use vfat, ext4, and xfs file systems

\item {} 
Mount and unmount CIFS and NFS network file systems

\item {} 
Extend existing logical volumes

\item {} 
Create and configure set-GID directories for collaboration

\item {} 
Create and manage Access Control Lists (ACLs)

\item {} 
Diagnose and correct file permission problems

\end{itemize}


\subsection{RHCSA Reqs - Deploy, configure, and maintain systems (1)}
\label{intro:rhcsa-reqs-deploy-configure-and-maintain-systems-1}\begin{itemize}
\item {} 
Configure networking and hostname resolution statically or dynamically

\item {} 
Schedule tasks using at and cron

\item {} 
Start and stop services and configure services to start automatically at boot

\item {} 
Configure systems to boot into a specific target automatically

\item {} 
Install Red Hat Enterprise Linux automatically using Kickstart

\item {} 
Configure a physical machine to host virtual guests

\item {} 
Install Red Hat Enterprise Linux systems as virtual guests

\end{itemize}


\subsection{RHCSA Reqs - Deploy, configure, and maintain systems (2)}
\label{intro:rhcsa-reqs-deploy-configure-and-maintain-systems-2}\begin{itemize}
\item {} 
Configure systems to launch virtual machines at boot

\item {} 
Configure network services to start automatically at boot

\item {} 
Configure a system to use time services

\item {} 
Install and update software packages from Red Hat Network, a remote repository, or from the local file system

\item {} 
Update the kernel package appropriately to ensure a bootable system

\item {} 
Modify the system bootloader

\end{itemize}


\subsection{RHCSA Reqs - Manage users and groups}
\label{intro:rhcsa-reqs-manage-users-and-groups}\begin{itemize}
\item {} 
Create, delete, and modify local user accounts

\item {} 
Change passwords and adjust password aging for local user accounts

\item {} 
Create, delete, and modify local groups and group memberships

\item {} 
Configure a system to use an existing authentication service for user and group information

\end{itemize}


\subsection{RHCSA Reqs - Manage security}
\label{intro:rhcsa-reqs-manage-security}\begin{itemize}
\item {} 
Configure firewall settings using firewall-config, firewall-cmd, or iptables

\item {} 
Configure key-based authentication for SSH

\item {} 
Set enforcing and permissive modes for SELinux

\item {} 
List and identify SELinux file and process context

\item {} 
Restore default file contexts

\item {} 
Use boolean settings to modify system SELinux settings

\item {} 
Diagnose and address routine SELinux policy violations

\end{itemize}


\section{RHCE Requirements (ex300)}
\label{intro:rhce-requirements-ex300}
Requirements for the RHCE exam can be found at -
\href{http://www.redhat.com/en/services/training/ex300-red-hat-certified-engineer-rhce-exam}{http://www.redhat.com/en/services/training/ex300-red-hat-certified-engineer-rhce-exam}


\subsection{RHCE Reqs - System configuration and management}
\label{intro:rhce-reqs-system-configuration-and-management}\begin{itemize}
\item {} 
Use network teaming or bonding to configure aggregated network links between two Red Hat Enterprise Linux systems

\item {} 
Configure IPv6 addresses and perform basic IPv6 troubleshooting

\item {} 
Route IP traffic and create static routes

\item {} 
Use firewalld and associated mechanisms such as rich rules, zones and custom rules, to implement packet filtering and configure network address translation (NAT)

\item {} 
Use /proc/sys and sysctl to modify and set kernel runtime parameters

\end{itemize}


\subsection{RHCE Reqs - System configuration and management}
\label{intro:id2}\begin{itemize}
\item {} 
Configure a system to authenticate using Kerberos

\item {} 
Configure a system as either an iSCSI target or initiator that persistently mounts an iSCSI target

\item {} 
Produce and deliver reports on system utilization (processor, memory, disk, and network)

\item {} 
Use shell scripting to automate system maintenance tasks

\end{itemize}


\subsection{RHCE Reqs - Network services}
\label{intro:rhce-reqs-network-services}\begin{itemize}
\item {} 
Install the packages needed to provide the service

\item {} 
Configure SELinux to support the service

\item {} 
Use SELinux port labeling to allow services to use non-standard ports

\item {} 
Configure the service to start when the system is booted

\item {} 
Configure the service for basic operation

\item {} 
Configure host-based and user-based security for the service

\end{itemize}


\subsection{RHCE Reqs - HTTP/HTTPS}
\label{intro:rhce-reqs-http-https}\begin{itemize}
\item {} 
Configure a virtual host

\item {} 
Configure private directories

\item {} 
Deploy a basic CGI application

\item {} 
Configure group-managed content

\item {} 
Configure TLS security

\end{itemize}


\subsection{RHCE Reqs - DNS}
\label{intro:rhce-reqs-dns}\begin{itemize}
\item {} 
Configure a caching-only name server

\item {} 
Troubleshoot DNS client issues

\end{itemize}


\subsection{RHCE Reqs - NFS}
\label{intro:rhce-reqs-nfs}\begin{itemize}
\item {} 
Provide network shares to specific clients

\item {} 
Provide network shares suitable for group collaboration

\item {} 
Use Kerberos to control access to NFS network shares

\end{itemize}


\subsection{RHCE Reqs - SMB}
\label{intro:rhce-reqs-smb}\begin{itemize}
\item {} 
Provide network shares to specific clients

\item {} 
Provide network shares suitable for group collaboration

\item {} 
Use Kerberos to authenticate access to shared directories

\end{itemize}


\subsection{RHCE Reqs - SMTP}
\label{intro:rhce-reqs-smtp}\begin{itemize}
\item {} 
Configure a system to forward all email to a central mail server

\end{itemize}


\subsection{RHCE Reqs - SSH}
\label{intro:rhce-reqs-ssh}\begin{itemize}
\item {} 
Configure key-based authentication

\item {} 
Configure additional options described in documentation

\end{itemize}


\subsection{RHCE Reqs - NTP}
\label{intro:rhce-reqs-ntp}\begin{itemize}
\item {} 
Synchronize time using other NTP peers

\end{itemize}


\subsection{RHCE Reqs - Database services}
\label{intro:rhce-reqs-database-services}\begin{itemize}
\item {} 
Install and configure MariaDB

\item {} 
Backup and restore a database

\item {} 
Create a simple database schema

\item {} 
Perform simple SQL queries against a database

\end{itemize}


\section{Workshop Prep}
\label{intro:workshop-prep}\begin{itemize}
\item {} 
Install RHEL on VM

\end{itemize}


\chapter{RHCSA Requirements (ex200)}
\label{rhcsa:rhcsa-requirements-ex200}\label{rhcsa::doc}
Requirements for the RHCSA exam can be found at-
\href{http://www.redhat.com/en/services/training/ex200-red-hat-certified-system-administrator-rhcsa-exam}{http://www.redhat.com/en/services/training/ex200-red-hat-certified-system-administrator-rhcsa-exam}


\section{Understand And Use Essential Tools}
\label{rhcsa:understand-and-use-essential-tools}

\subsection{Shell Commands}
\label{rhcsa:shell-commands}
Access a shell prompt and issue commands with correct syntax

\begin{Verbatim}[commandchars=\\\{\}]
\PYGZgt{} This is a sample of some code.
\end{Verbatim}


\subsection{Redirection}
\label{rhcsa:redirection}
Use input-output redirection (\textgreater{}, \textgreater{}\textgreater{}, \textbar{}, 2\textgreater{}, etc.)


\subsection{Grep/Regex}
\label{rhcsa:grep-regex}
Use grep and regular expressions to analyze text


\subsection{SSH}
\label{rhcsa:ssh}
Access remote systems using ssh


\subsection{Switch Users}
\label{rhcsa:switch-users}
Log in and switch users in multiuser targets


\subsection{Archiving}
\label{rhcsa:archiving}
To archive/unarchive directories:

\begin{Verbatim}[commandchars=\\\{\}]
\PYG{n+nv}{\PYGZdl{} }tar \PYGZhy{}\PYGZhy{}selinux \PYGZhy{}czvf \PYGZlt{}directory.tgz\PYGZgt{} \PYGZlt{}directory\PYGZgt{}
\PYG{n+nv}{\PYGZdl{} }tar xzvf \PYGZlt{}directory.tgz\PYGZgt{}

\PYG{n+nv}{\PYGZdl{} }star \PYGZhy{}xattr \PYGZhy{}H\PYG{o}{=}exustar \PYGZhy{}c \PYGZhy{}f\PYG{o}{=}\PYGZlt{}directory.star\PYGZgt{} \PYGZlt{}directory\PYGZgt{}
\PYG{n+nv}{\PYGZdl{} }star \PYGZhy{}x \PYGZhy{}f\PYG{o}{=}\PYGZlt{}directory.star\PYGZgt{}
\end{Verbatim}


\subsection{Compress}
\label{rhcsa:compress}
To compress/uncompress a file:

\begin{Verbatim}[commandchars=\\\{\}]
\PYG{n+nv}{\PYGZdl{} }gzip \PYGZlt{}file\PYGZgt{}
\PYG{n+nv}{\PYGZdl{} }gunzip \PYGZlt{}file.gz\PYGZgt{}

\PYG{n+nv}{\PYGZdl{} }bzip2 \PYGZlt{}file\PYGZgt{}
\PYG{n+nv}{\PYGZdl{} }bunzip2 \PYGZlt{}file.bz2\PYGZgt{}
\end{Verbatim}


\subsection{Text Files}
\label{rhcsa:text-files}
Create and edit text files


\subsection{Files}
\label{rhcsa:files}
Create, delete, copy, and move files and directories


\subsection{Links}
\label{rhcsa:links}
Create hard and soft links


\subsection{Permissions}
\label{rhcsa:permissions}
List, set, and change standard ugo/rwx permissions


\subsection{Documentation}
\label{rhcsa:documentation}
Locate, read, and use system documentation including man, info, and files in /usr/share/doc


\section{Operate Running Systems}
\label{rhcsa:operate-running-systems}

\subsection{Normal Booting}
\label{rhcsa:normal-booting}
Boot, reboot, and shut down a system normally


\subsection{Boot Targets}
\label{rhcsa:boot-targets}
Boot systems into different targets manually


\subsection{Interrupt Boot}
\label{rhcsa:interrupt-boot}
Interrupt the boot process in order to gain access to a system


\subsection{Processes}
\label{rhcsa:processes}
Identify CPU/memory intensive processes, adjust process priority with renice, and kill processes


\subsection{Logs}
\label{rhcsa:logs}
Locate and interpret system log files and journals


\subsection{RHCSA Reqs - Operate running systems (2)}
\label{rhcsa:rhcsa-reqs-operate-running-systems-2}\begin{itemize}
\item {} 
Access a virtual machine's console

\item {} 
Start and stop virtual machines

\item {} 
Start, stop, and check the status of network services

\item {} 
Securely transfer files between systems

\end{itemize}


\subsection{RHCSA Reqs - Configure local storage}
\label{rhcsa:rhcsa-reqs-configure-local-storage}\begin{itemize}
\item {} 
List, create, delete partitions on MBR and GPT disks

\item {} 
Create and remove physical volumes, assign physical volumes to volume groups, and create and delete logical volumes

\item {} 
Configure systems to mount file systems at boot by Universally Unique ID (UUID) or label

\item {} 
Add new partitions and logical volumes, and swap to a system non-destructively

\end{itemize}


\subsection{RHCSA Reqs - Create and configure file systems}
\label{rhcsa:rhcsa-reqs-create-and-configure-file-systems}\begin{itemize}
\item {} 
Create, mount, unmount, and use vfat, ext4, and xfs file systems

\item {} 
Mount and unmount CIFS and NFS network file systems

\item {} 
Extend existing logical volumes

\item {} 
Create and configure set-GID directories for collaboration

\item {} 
Create and manage Access Control Lists (ACLs)

\item {} 
Diagnose and correct file permission problems

\end{itemize}


\subsection{RHCSA Reqs - Deploy, configure, and maintain systems (1)}
\label{rhcsa:rhcsa-reqs-deploy-configure-and-maintain-systems-1}\begin{itemize}
\item {} 
Configure networking and hostname resolution statically or dynamically

\item {} 
Schedule tasks using at and cron

\item {} 
Start and stop services and configure services to start automatically at boot

\item {} 
Configure systems to boot into a specific target automatically

\item {} 
Install Red Hat Enterprise Linux automatically using Kickstart

\item {} 
Configure a physical machine to host virtual guests

\item {} 
Install Red Hat Enterprise Linux systems as virtual guests

\end{itemize}


\subsection{RHCSA Reqs - Deploy, configure, and maintain systems (2)}
\label{rhcsa:rhcsa-reqs-deploy-configure-and-maintain-systems-2}\begin{itemize}
\item {} 
Configure systems to launch virtual machines at boot

\item {} 
Configure network services to start automatically at boot

\item {} 
Configure a system to use time services

\item {} 
Install and update software packages from Red Hat Network, a remote repository, or from the local file system

\item {} 
Update the kernel package appropriately to ensure a bootable system

\item {} 
Modify the system bootloader

\end{itemize}


\subsection{RHCSA Reqs - Manage users and groups}
\label{rhcsa:rhcsa-reqs-manage-users-and-groups}\begin{itemize}
\item {} 
Create, delete, and modify local user accounts

\item {} 
Change passwords and adjust password aging for local user accounts

\item {} 
Create, delete, and modify local groups and group memberships

\item {} 
Configure a system to use an existing authentication service for user and group information

\end{itemize}


\subsection{RHCSA Reqs - Manage security}
\label{rhcsa:rhcsa-reqs-manage-security}\begin{itemize}
\item {} 
Configure firewall settings using firewall-config, firewall-cmd, or iptables

\item {} 
Configure key-based authentication for SSH

\item {} 
Set enforcing and permissive modes for SELinux

\item {} 
List and identify SELinux file and process context

\item {} 
Restore default file contexts

\item {} 
Use boolean settings to modify system SELinux settings

\item {} 
Diagnose and address routine SELinux policy violations

\end{itemize}


\chapter{RHCE Review}
\label{rhce:rhce-review}\label{rhce::doc}

\section{RHCE Requirements (ex300)}
\label{rhce:rhce-requirements-ex300}
Requirements for the RHCE exam can be found at -
\href{http://www.redhat.com/en/services/training/ex300-red-hat-certified-engineer-rhce-exam}{http://www.redhat.com/en/services/training/ex300-red-hat-certified-engineer-rhce-exam}


\subsection{RHCE Reqs - System configuration and management (1)}
\label{rhce:rhce-reqs-system-configuration-and-management-1}\begin{itemize}
\item {} 
Use network teaming or bonding to configure aggregated network links between two Red Hat Enterprise Linux systems

\item {} 
Configure IPv6 addresses and perform basic IPv6 troubleshooting

\item {} 
Route IP traffic and create static routes

\item {} 
Use firewalld and associated mechanisms such as rich rules, zones and custom rules, to implement packet filtering and configure network address translation (NAT)

\item {} 
Use /proc/sys and sysctl to modify and set kernel runtime parameters

\end{itemize}


\subsection{RHCE Reqs - System configuration and management (2)}
\label{rhce:rhce-reqs-system-configuration-and-management-2}\begin{itemize}
\item {} 
Configure a system to authenticate using Kerberos

\item {} 
Configure a system as either an iSCSI target or initiator that persistently mounts an iSCSI target

\item {} 
Produce and deliver reports on system utilization (processor, memory, disk, and network)

\item {} 
Use shell scripting to automate system maintenance tasks

\end{itemize}


\subsection{RHCE Reqs - Network services}
\label{rhce:rhce-reqs-network-services}\begin{itemize}
\item {} 
Install the packages needed to provide the service

\item {} 
Configure SELinux to support the service

\item {} 
Use SELinux port labeling to allow services to use non-standard ports

\item {} 
Configure the service to start when the system is booted

\item {} 
Configure the service for basic operation

\item {} 
Configure host-based and user-based security for the service

\end{itemize}


\subsection{RHCE Reqs - HTTP/HTTPS}
\label{rhce:rhce-reqs-http-https}\begin{itemize}
\item {} 
Configure a virtual host

\item {} 
Configure private directories

\item {} 
Deploy a basic CGI application

\item {} 
Configure group-managed content

\item {} 
Configure TLS security

\end{itemize}


\subsection{RHCE Reqs - DNS}
\label{rhce:rhce-reqs-dns}\begin{itemize}
\item {} 
Configure a caching-only name server

\item {} 
Troubleshoot DNS client issues

\end{itemize}


\subsection{RHCE Reqs - NFS}
\label{rhce:rhce-reqs-nfs}\begin{itemize}
\item {} 
Provide network shares to specific clients

\item {} 
Provide network shares suitable for group collaboration

\item {} 
Use Kerberos to control access to NFS network shares

\end{itemize}


\subsection{RHCE Reqs - SMB}
\label{rhce:rhce-reqs-smb}\begin{itemize}
\item {} 
Provide network shares to specific clients

\item {} 
Provide network shares suitable for group collaboration

\item {} 
Use Kerberos to authenticate access to shared directories

\end{itemize}


\subsection{RHCE Reqs - SMTP}
\label{rhce:rhce-reqs-smtp}\begin{itemize}
\item {} 
Configure a system to forward all email to a central mail server

\end{itemize}


\subsection{RHCE Reqs - SSH}
\label{rhce:rhce-reqs-ssh}\begin{itemize}
\item {} 
Configure key-based authentication

\item {} 
Configure additional options described in documentation

\end{itemize}


\subsection{RHCE Reqs - NTP}
\label{rhce:rhce-reqs-ntp}\begin{itemize}
\item {} 
Synchronize time using other NTP peers

\end{itemize}


\subsection{RHCE Reqs - Database services}
\label{rhce:rhce-reqs-database-services}\begin{itemize}
\item {} 
Install and configure MariaDB

\item {} 
Backup and restore a database

\item {} 
Create a simple database schema

\item {} 
Perform simple SQL queries against a database

\end{itemize}



\renewcommand{\indexname}{Index}
\printindex
\end{document}
